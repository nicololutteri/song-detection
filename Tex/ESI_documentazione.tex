% !TEX encoding = UTF-8

\documentclass[a4paper, 12pt]{article}
\usepackage[T1]{fontenc}
\usepackage[utf8]{inputenc}
\usepackage[italian]{babel}

\begin{document}

\title{%
    \normalfont \normalsize %
    \Large\textsc{Università degli Studi di Verona} \\ %
    {Dipartimento di Informatica} \\ [25pt] %
    \rule{\linewidth}{0.5pt} \\ [0.4cm] %
    \huge Elaborazioni di Segnali ed Immagini \\ %
    \Large Sviluppo di un algoritmo per il riconscimento delle canzoni
    \rule{\linewidth}{2pt} \\ [0.5cm] %
}
\author{Nicolò Lutteri \and Damian Mastroiacovo \and Luigi Capogrosso}
\date{\today}
\maketitle

\tableofcontents

\newpage

\section{Specifiche del progetto}
\begin{itemize}
\item \textbf{Obiettivi}: in generale, l'idea è quella di sviluppare il codice visto a lezione sul tema in oggetto, ed in particolare si articola nei seguenti sotto-obiettivi:
\begin{enumerate}
\item Individuare un numero di casi di studio elevato (20 almeno) dove applicare, variando i parametri del codice visto a lezione;
\item Grazie all'analisi di cui al punto precedente, l'idea è di capire quali condizioni di acquisizione non permettono una buona accuratezza (= numero di casi giusti/numero di casi totali);
\item Capire come vari l'andamento dell'accuratezza al variare della lunghezza del segmento di test.
\end{enumerate}
\item Come prendere il massimo dei voti:
\begin{enumerate}
\item Riuscendo a sviluppare tutti i sotto obiettivi;
\item Considerando esempi di canzoni diverse tra loro e non troppo simili a quelle viste in aula.
\end{enumerate}
\end{itemize}

\section{Scopo di questo documento}
Lo scopo che si prefigge questo docuemento è quello di spiegare come il progetto è stato implementato, in particolare, la finalità è quella di mostrare come sono stati sviluppati tutti i sotto obiettivi illustrando tutti i test svolti.

\section{Sotto-obiettivo 1}
\emph{Individuare un numero di casi di studio elevato (20 almeno) dove applicare, variando i parametri del codice visto a lezione}. \\

Per la creazione dei casi d'uso, la nostra gestione è stata la seguente:
\begin{itemize}
\item Nella cartella \verb|Rumore/| abbiamo inserito 10 file \verb|.mp3| che simulano un disturbo (applausi, bambino che piange, ambulanza, ecc\dots{});
\item Nella cartella \verb|Libreria/| abbiamo inserito 20 file \verb|.mp3| che risultano, invece, essere canzoni di differenti generi (rock, pop, jaz, latino, ecc\dots{}).
\end{itemize}
Ogni audio contenuto in \verb|Rumore/| è stato sommato con tutte le canzoni contenute in \verb|Libreria/|, generando così un nuovo segnale, poi, tagliato a 1 2 3 4 5 6 7 8 9 10 secondi. \\
In totale i casi creati dovrebbero quindi essere: $10*20*10=2000$. \\
Nel nostro caso specifico però, questi, sono esattamente \textbf{780}, poiché, nella funzione \verb|SommaSegnali.m| abbiamo fatto dei controlli preventivi prima della somma di \verb|S1| ed \verb|S2|.
Tutti i file sono stati poi salvati nella cartella \verb|Casi/| atraverso codice \verb|Matlab|.

\section{Sotto-obiettivo 2}
\emph{Grazie all'analisi di cui al punto precedente, l'idea è di capire quali condizioni di acquisizione non permettono una buona accuratezza (= numero di casi giusti/numero di casi totali)}. \\

I risultati da noi ottenuti sono:
\begin{itemize}
\item \textbf{Lunghezza 1}:
\begin{itemize}
\item Totali: 180
\item Giusti: 54
\item Sbagliati: 126	
\item Rapporto: 30\%
\end{itemize}
\item \textbf{Lunghezza 2}:
\begin{itemize}
\item Totali: 180
\item Giusti: 88
\item Sbagliati: 92
\item Rapporto: 48\%
\end{itemize}
\item \textbf{Lunghezza 3}:
\begin{itemize}
\item Totali: 100
\item Giusti: 68
\item Sbagliati: 32
\item Rapporto: 68\%
\end{itemize}

\newpage

\item \textbf{Lunghezza 4}:
\begin{itemize}
\item Totali: 80
\item Giusti: 59
\item Sbagliati: 21
\item Rapporto: 73\%
\end{itemize}
\item \textbf{Lunghezza 5}:
\begin{itemize}
\item Totali: 80
\item Giusti: 59
\item Sbagliati: 21
\item Rapporto: 73\%
\end{itemize}
\item \textbf{Lunghezza 6}:
\begin{itemize}
\item Totali: 40
\item Giusti: 32
\item Sbagliati: 8
\item Rapporto: 80\%
\end{itemize}
\item \textbf{Lunghezza 7}:
\begin{itemize}
\item Totali: 40
\item Giusti: 32
\item Sbagliati: 8
\item Rapporto: 80\%
\end{itemize}
\item \textbf{Lunghezza 8}:
\begin{itemize}
\item Totali: 40
\item Giusti: 34 
\item Sbagliati: 6
\item Rapporto: 85\%
\end{itemize}
\item \textbf{Lunghezza 9}:
\begin{itemize}
\item Totali: 20
\item Giusti: 16
\item Sbagliati: 4
\item Rapporto: 80\%
\end{itemize}
\item \textbf{Lunghezza 10}:
\begin{itemize}
\item Totali: 20
\item Giusti: 16
\item Sbagliati: 4
\item Rapporto: 80\%
\end{itemize}
\end{itemize}

\section{Sotto-obiettivo 3}
\emph{Capire come vari l'andamento dell'accuratezza al variare della lunghezza del segmento di test}. \\

I nostri test mostrano che l'andamento dell'accuratezza al variare della lunghezza del segmento aumenta. Difatti, con una lunghezza del segmento pari a \textbf{2 secondi}, abbiamo una percentaule di casi corretti del \textbf{48\%}, mentre, con una lunghezza pari a \textbf{10 secondi} il rapporto risulta essere del \textbf{80\%}. \\

Questo a dimostrazione del fatto che, l'accuratezza aumenta all'aumentare della lunghezza del segmento di test, risultato in linea con ciò che noi ci aspettavamo.

\end{document}